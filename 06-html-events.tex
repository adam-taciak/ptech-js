\section{Zdarzenia HTML}

\begin{frame}[fragile]
  \frametitle{Zdarzenia}
  \framesubtitle{Wprowadzenie}

  \textbf{HTML} posiada zestaw zdarzeń które mogą wywoływać kod \textbf{JavaScript}

  \begin{itemize}
    \item zdarzenia okna (\verb|onload|, \verb|onresize|)
    \item zdarzenia formularzy (\verb|onfocus|, \verb|onsubmit|)
    \item zdarzenia klawiatury (\verb|onkeydown|, \verb|onkeyup|)
    \item zdarzenia myszy (\verb|onclick|, \verb|onwheel|)
    \item zdarzenia przenoszenia (\verb|ondrag|, \verb|ondrop|)
    \item zdarzenia schowka (\verb|oncopy|, \verb|onpaste|)
  \end{itemize}

\end{frame}


\begin{frame}[fragile]
  \frametitle{Zdarzenia}
  \framesubtitle{Wprowadzenie}

  Zdarzenia zezwalają na wykonywanie kodu \textbf{JavaScript} w interakcji z użytkownikiem:

  \begin{itemize}
    \item (Z) po pełnym załadowaniu strony, (JS) wyświetl komunikat
    \item (Z) kliknięcie w przycisk, (JS) zmienia kolor strony
    \item (Z) po najechaniu kursorem na zdjęcie, (JS) powiększ je
  \end{itemize}
\end{frame}


\begin{frame}[fragile]
  \frametitle{Zdarzenia}
  \framesubtitle{Wprowadzenie}

  Przykład użycia:

  \begin{minted}{html}
<button onclick="alert('Witaj świecie')">
</button>
  \end{minted}

\end{frame}

