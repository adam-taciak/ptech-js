\section{Typy danych}

\begin{frame}[fragile]
  \frametitle{Typy danych}
  \framesubtitle{Wprowadzenie}

  Wartość w języku \textbf{JavaScript} zawsze ma określony typ. 

  \begin{itemize}
    \item \verb|number| --- liczby
    \item \verb|bigInt| --- duże liczby
    \item \verb|string| --- ciągi znakowe
    \item \verb|boolean| --- wartość prawda/fałsz
    \item wartosć \verb|null| --- wartość pusta
    \item wartość \verb|undefined| --- wartość niezainicjalizowana
    \item \verb|object| --- objekt
  \end{itemize}

\end{frame}

\begin{frame}[fragile]
  \frametitle{Typy danych}
  \framesubtitle{number}

 Typ \verb|number| przechowuje liczbę:

  \begin{minted}{js}
let liczba_a = 123;
let liczba_b = 12.3; 
  \end{minted}

\end{frame}


\begin{frame}[fragile]
  \frametitle{Typy danych}
  \framesubtitle{number}

 Na typie \verb|number| można wykonywać operacje matematyczne:

  \begin{minted}{js}
let a = 10 + 2;
let b = 20 - 2;
let c = 5 * 5;
let d = 16 / 8;
  \end{minted}

\end{frame}


\begin{frame}[fragile]
  \frametitle{Typy danych}
  \framesubtitle{string}

 Typ \verb|string| przechowuje łańcuch znakowy:

  \begin{minted}{js}
let napis_a = 'Ala ma kota';
let napis_b = "Ala ma psa";
let napis_c = `Ala ma chomika`;
  \end{minted}

\end{frame}


\begin{frame}[fragile]
  \frametitle{Typy danych}
  \framesubtitle{string}

 Na typie \verb|string| można wykonywać operację konkatenacji (łączenia):

  \begin{minted}{js}
let napis = 'Ala ' + 'ma' + ' kota';
  \end{minted}

\end{frame}

\begin{frame}[fragile]
  \frametitle{Typy danych}
  \framesubtitle{boolean}

 Typ \verb|boolean| przechowuje informację prawdwa/fałsz:

  \begin{minted}{js}
let prawda = true;
let falsz = false;
  \end{minted}

\end{frame}


\begin{frame}[fragile]
  \frametitle{Typy danych}
  \framesubtitle{string}

 Typ \verb|boolean| najczęściej będziemy używać przy konstrukcjach warunkowych:

  \begin{minted}{js}
let czyPelnoletni = false;
if (czyPelnoletni) {
  alert('Użytkownik jest pełnoletni');
}
  \end{minted}

\end{frame}


\begin{frame}[fragile]
  \frametitle{Typy danych}
  \framesubtitle{null}

 Wartość \verb|null| oznacza nic, brak danych:

  \begin{minted}{js}
let napis = null;
  \end{minted}

  Zmienna \verb|napis| przechowuje informację o braku informacji.

\end{frame}


\begin{frame}[fragile]
  \frametitle{Typy danych}
  \framesubtitle{undefined}

 Wartość \verb|undefined| oznacza, iż zmienna pozostaje niezainicjalizowana:

  \begin{minted}{js}
let napis;
console.log(napis);
  \end{minted}

  Zmienna \verb|napis| została zadeklarowana, jednak jej początkowa wartość pozostaje nieokreślona.

\end{frame}


\begin{frame}[fragile]
  \frametitle{Typy danych}
  \framesubtitle{object}

 Typ \verb|object| jest typem złożonym

  \begin{minted}{js}
let samochod = {
  marka: 'Porsche',
  model: '911 Turbo',
  predkoscMax: 210
}
  \end{minted}

  Zmienna \verb|samochod| reprezentuje obiekt zbudowany z 3 pól (typów prostych).

\end{frame}
