\section{Stałe}

\begin{frame}[fragile]
  \frametitle{Stałe}
  \framesubtitle{Wprowadzenie}

  Dane możemy przechowywać nie tylko w zmiennych, ale również w stałych:

  \begin{minted}{js}
let zmienna;
zmienna = 10;

const stala = 10;
  \end{minted}

\end{frame}


\begin{frame}[fragile]
  \frametitle{Stałe}
  \framesubtitle{Wprowadzenie}

  Stałe tworzy się podobnie jak zmianne:

  \begin{itemize}
    \item słowo kluczowe \verb|const|
    \item nazwa stałej
    \item należy od razu przypisać wartość
  \end{itemize}

\end{frame}


\begin{frame}[fragile]
  \frametitle{Stałe}
  \framesubtitle{Wprowadzenie}

  Stałe:

  \begin{itemize}
    \item gwarantują niezmienialność danych
    \item należy inicjalizować w chwili tworzenia
    \item po inicjalizacji, nie można zmienić wartości
  \end{itemize}

\end{frame}
