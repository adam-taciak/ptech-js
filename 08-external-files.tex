\section{Osadzanie JavaScript}

\begin{frame}[fragile]
  \frametitle{Osadzanie JavaScript}
  \framesubtitle{Wprowadzenie}

  Kod \textbf{JavaScript} możemy osadzać w kodzie \textbf{HTML} na kilka sposobów:

  \begin{itemize}
    \item wewnątrz zdarzeń \verb|onclick=""|
    \item wewnątrz tagu \verb|<script></script>|
    \item w zewnętrzych plikach \verb|<script src="nazwa.js"></script>|
  \end{itemize}

\end{frame}


\begin{frame}[fragile]
  \frametitle{Osadzanie JavaScript}
  \framesubtitle{Wprowadzenie}

  Podobnie jak z \textbf{CSS}, kod \textbf{JavaScrip} warto umieścić w zewnętrznym pliku:

  \begin{itemize}
    \item logiczne rozdzielenie \textbf{HTML}, \textbf{CSS}, \textbf{JS}
    \item większa czytelność \textbf{HTML}, \textbf{JavaScript}
    \item uproszczenie kody, podział między pliki
  \end{itemize}

\end{frame}


\begin{frame}[fragile]
  \frametitle{Osadzanie JavaScript}
  \framesubtitle{Przypomnienie}

  Przypomnienie --- zdarzenia:

  \begin{minted}{html}
<button onclick="alert('Witaj!');">
  Kliknij mnie!
</button>
  \end{minted}

\end{frame}


\begin{frame}[fragile]
  \frametitle{Osadzanie JavaScript}
  \framesubtitle{Przypomnienie}

  Przypomnienie --- \verb|<script></script>|:

  \begin{minted}{html}
<head>
  <script>
    alert('Witaj świecie');
  </script>
</head>

<body>
  <script>
    alert('Witaj świecie');
  </script>
</body>
  \end{minted}

\end{frame}


\begin{frame}[fragile]
  \frametitle{Osadzanie JavaScript}
  \framesubtitle{JS z pliku}

  Podział na pliki:

  Kod \textbf{JavaScript} w pliku \verb|nazwa.js|
  \begin{minted}{js}
alert('Witaj!');
  \end{minted}

  Kod \textbf{HTML}
  \begin{minted}{html}
<head>
  <script src="nazwa.js"></script>
</head>
  \end{minted}

\end{frame}


\begin{frame}[fragile]
  \frametitle{Osadzanie JavaScript}
  \framesubtitle{JS z pliku}

  Bardzo dobrą praktyką jest dodawanie atrybutu \textbf{type}.

  \begin{itemize}
    \item \textbf{type} informuje przeglądarkę o oczekiwanej zawartości
    \item wartość \textbf{application/javascript} oznacza skrypt w \textbf{JavaScript}
  \end{itemize}

  \begin{minted}{html}
<head>
  <script src="nazwa.js" type="application/javascript">
  </script>
</head>
  \end{minted}

\end{frame}


\begin{frame}[fragile]
  \frametitle{Osadzanie JavaScript}
  \framesubtitle{JS z pliku}

  W chwili załadowania pliku tagiem:

  \begin{minted}{html}
<script src="plik.js"
  type="application/javascript">
</script>
  \end{minted}

  Kod staje się dostępny w całej aplikacji.

\end{frame}


\begin{frame}[fragile]
  \frametitle{Osadzanie JavaScript}
  \framesubtitle{JS z pliku}

  \textbf{JavaScript} plik \verb|plik.js|:

  \begin{minted}{js}
let imie = 'Adam';
let wiek = 34;
  \end{minted}

  \textbf{HTML}:

  \begin{minted}{html}
<head>
  <script src="plik.js" type="application/javascript">
  </script>
</head>
<body>
  <button onclick="alert('Witaj' + ' ' + imie);">
  </button>
</body>
  \end{minted}

\end{frame}
