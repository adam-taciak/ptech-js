\section{Funkcje}

\begin{frame}[fragile]
  \frametitle{Funkcje}
  \framesubtitle{Wprowadzenie}

  Funkcji używamy do podziału programu na mniejsze logiczne jednostki:

  \begin{itemize}
    \item zwiększenie czytelności programu
    \item unikanie duplikacji kodu
    \item łatwiejsze utrzymanie
  \end{itemize}

\end{frame}


\begin{frame}[fragile]
  \frametitle{Funkcje}
  \framesubtitle{Wywoływanie funkcji}

  Wywoływanie funkcjiw \textbf{JavaScript} odbywa się przez jej nazwę z dodaniem nawiasów:

  \begin{minted}{js}
alert( );
  \end{minted}

  Uwaga: jeżli pominiemy \verb|()| odwołamy się do obiektu funkcji!

\end{frame}


\begin{frame}[fragile]
  \frametitle{Funkcje}
  \framesubtitle{Wywoływanie funkcji}

  Do funkcji argumenty przekazujemy wewnątrz nawiasów:

  \begin{minted}{js}
alert('Ala ma kota');
  \end{minted}

  Jeżeli funkcja nie przyjmuje argumentów, to nadal należy napisać \verb|()|.

\end{frame}


\begin{frame}[fragile]
  \frametitle{Funkcje}
  \framesubtitle{Tworzenie funkcji}

  Tworzenie nowej funkcji odbywa się przez słowo kluczowe \verb|function|:

  \begin{minted}{js}
function nazwa() {

}
  \end{minted}

\end{frame}


\begin{frame}[fragile]
  \frametitle{Funkcje}
  \framesubtitle{Tworzenie funkcji}

  Funkcja składa się z:
  
  \begin{itemize}
    \item słowa \verb|function|
    \item swojej nazwy
    \item listy argumentów w \verb|()|
    \item ciałem funkcji wewnątrz \verb|{}|
  \end{itemize}

  \begin{minted}{js}
function nazwa() {

}
  \end{minted}

\end{frame}


\begin{frame}[fragile]
  \frametitle{Funkcje}
  \framesubtitle{Tworzenie funkcji}

  \begin{minted}{js}
function powiatanie_adam() {
  alert('Witaj Adam!');
}

function powitanie_ewa() {
  alert('Witaj Ewa!');
}

witaj_adam();
witaj_ewa();
  \end{minted}

\end{frame}


\begin{frame}[fragile]
  \frametitle{Funkcje}
  \framesubtitle{Tworzenie funkcji}

  Funkcja może zwracać wartość, służy do tego słowo kluczowe \verb|return|:

  \begin{minted}{js}
function moj_wiek() {
  return 17;
}

let wiek = moj_wiek();
  \end{minted}

\end{frame}


\begin{frame}[fragile]
  \frametitle{Funkcje}
  \framesubtitle{Tworzenie funkcji}

  Funkcja może przyjmować argumenty:

  \begin{minted}{js}
function powitanie(imie) {
  alert('Witaj' + ' ' + imie);
}
  \end{minted}

\end{frame}


\begin{frame}[fragile]
  \frametitle{Funkcje}
  \framesubtitle{Tworzenie funkcji}

  Funkcja może przyjmować argumenty i zwracać wartość:

  \begin{minted}{js}
function suma(a, b) {
  let wynik = a + b;
  return wynik;
}

let obliczenie = suma(10, 20);
  \end{minted}

\end{frame}


\begin{frame}[fragile]
  \frametitle{Funkcje}
  \framesubtitle{Funkcja a zasięg zmiennych}

  Funkcje posiadają własny zasięg (ang. scope) zmiennych:

  \begin{minted}{js}
function nazwa() {
  let zmienna = 10;
}

alert(zmienna); // undefined
  \end{minted}

  Zmienne definiowane wewnątrz funkcji, nie są dostępne na zewnątrz.

\end{frame}


\begin{frame}[fragile]
  \frametitle{Funkcje}
  \framesubtitle{Zastosowanie}

  Do czego ta wiedza może się przydać? Od teraz możliwe będzie utrzymanie czystości i czytelności kodu:

  \begingroup
    \footnotesize

  \begin{minted}{js}
function skopiuj() {
  let wejscie = document.getElementById('wejscie');
  let wyjscie = document.getElementById('wyjscie');
  wyjscie.input = wejscie.input;
}
  \end{minted}

  \begin{minted}{html}
<head>
  <script src="skrypt.js"></script>
</head>
<body>
  <input id="wejscie">
  <input id="wyjscie">
  <button onclick="skopiuj()"></button>
</body>
  \end{minted}

  \endgroup

\end{frame}


\begin{frame}[fragile]
  \frametitle{Funkcje}
  \framesubtitle{Funkcje strzałkowe}

  W języku \textbf{JavScript} istnieje jeszcze inny sposób tworzenia funkcji:

  \begin{minted}{js}
const witaj = () => alert('Witaj');
  \end{minted}

  Co odpowiada

  \begin{minted}{js}
function witaj() {
  return alert('Witaj');
}
  \end{minted}

\end{frame}


\begin{frame}[fragile]
  \frametitle{Funkcje}
  \framesubtitle{Funkcje strzałkowe}

  Są to \textbf{funkcje strzałkowe} (ang. arrow functions), w innych językach nazywane również funkcjami anonimowymi.

  \begin{itemize}
    \item są lukrem składniowym
    \item skracają zapis
    \item mogą zwiększać czytelność kodu
    \item nie posiadają swojej nazwy
    \item zawsze zwracają wartość
  \end{itemize}

\end{frame}
